\chapter{Teória strojového učenia I}

Chceme sa naučiť na základe nejakých vstupných dát $x$ predikovať $y$.
Môžeme si to predstaviť tak, že príroda vie poskytovať pozorovania,
každé v tvare dvojice $(x, y)$. Dostali sme od nej sadu $t$ pozorovaní,
na základe ktorých chceme navrhnúť nejakú funkciu $h$, ktorá predpovedá
$y$ na základe $x$. Dobrá funkcia je taká, ktorá je schopná
\emph{zovšeobecňovať}, teda sa jej ``dobré darí'' aj na dátach mimo
trénovacej množiny. Proces, ktorým $h$ zostrojíme, si môžeme predtaviť
ako algoritmus, ktorý berie ako vstupy trénovacie dáta a vráti nám
funkciu.




\section{Matematický model}

Z matematického hľadiska, prírodu vieme formalizovať ako
pravdepodobnostnú distribúciu $P$. Množinu všetkých možných
$x$ označíme $X$, množinu možných $y$ označíme $Y$.

V tejto časti sa nebudeme zaoberať výpočtovou stránkou strojového
učenia, od detailov ako časová zložitosť, ..., abstrahujeme. Algoritmus
si teda predstavíme iba ako niečo, čo vezme ako vstup trénovacie dáta
$(x_1, y_1), \ldots, (x_t, y_t)$ a na výstup vráti funkciu
$h : X \to Y$. Túto funkciu budeme volať \emph{hypotéza}. Množinu
všetkých možných funkcii, ktoré môže náš algoritmus vrátiť,
budeme volať \emph{množina hypotéz} a značiť ho $H$.

\paragraph{Chyba hypotézy.}
Ako vyjadriť mieru toho, že sa funkcii ``dobre darí''? Spravíme tak
pomocou \emph{chybovej funkcie} $\err : Y \times Y \to \reals^+$,
ktorej význam je nasledovný: $\err(y, y')$ vyjadruje, ako veľmi
sa od seba líšia $y$ a $y'$. Pomocou tejto funkcie vieme odmerať
priemernú chybu hypotézy $h$, ktorú budeme tiež označovať $\err$,
nasledovne:
$$\err(h) = \E_{x, y} \left[ \err(h(x), y) \right]$$
Pod $\E_{x, y}$ sa rozumie stredná hodnota cez $(x, y)$
z pravdepodobnostnej distribúcie $P$, teda $(x, y) \sim P$.
Pri klasifikácii sa zvykne používať chybová funkcia
$$
  \err(y, y') = \left\{
    \begin{array}{ll}
      0, & \text{ak}\ y = y' \\
      1, & \text{inak}
    \end{array}
  \right.
$$
a potom zrejme
$$\E_{x, y} \left[ \err(h(x), y) \right] = \prob_{x, y} \left( h(x) \neq y \right).$$
Pri regresii máme viacero možností, bežné voľby sú kvadratická chyba
$(y - y')^2$ a absolútna chyba $|y - y'|$.

\paragraph{Chyba algoritmu.}
Ako vyjadriť chybu celého učiaceho algoritmu? Uvedomme si, že výstup
algoritmu je závislý od trénovacích dát $T = \{(x_1, y_1), \ldots, (x_t, y_t)\}$,
ktoré dostane. Takže výstupná funkcia je od nich závislá, budeme ju
označovať $\hat{h}$. Potom priemerná chyba algoritmu (alebo inak
\emph{priemerná chyba priemernej hypotézy}), braná cez všetky možné
vzorky trénovacích dát, je rovná
$$\E_T \left[ \err(\hat{h}) \right] = \E_T \left[ \E_{x,y} \left[ \err(\hat{h}(x), y) \right] \right].$$
Pod $\E_T$ sa rozumie stredná hodnota cez všetky možné $t$-tice
trénovacích dát $T$, brané nezávisle z pravdepodobnostnej
distribúcie $P$.

\paragraph{Trénovacia chyba.}
Pri vyššie uvedených chybách sme vždy merali vzhľadom na skutočnú
distribúciu $P$. Môže nás ale zaujímať, aká je priemerná chyba
hypotézy na trénovacích dátach $T$. Túto chybu budeme označovať
$\err_T(h)$, a vypočítame ju ako
$$\err_T(h) = \E_{x_i, y_i} \left[ \err(h(x_i), y_i) \right] = \frac{1}{t} \cdot \sum_{i=1}^t \err(h(x_i), y_i).$$

Priemerná trénovacia chyba z pohľadu algoritmu bude
$$\E_T \left[ \err_T(\hat{h}) \right].$$

V nasledujúcom texte budeme vynechávať premenné, cez ktoré prebiehajú
stredné hodnoty, všade tam, kde budú zrejmé z kontextu.




\section{Analýza veľkostí chýb}

V tejto časti sa podrobnejšie pozrieme na to, ako závisia vyššie
uvedené štatistiky (tj. priemerná testovacia a trénovacia chyba
priemernej hypotézy) od veľkosti trénovacej množiny $T$ a od veľkosti
množiny hypotéz $H$.

V celej časti budeme predpokladať, že úloha je regresného
charakteru a chyba sa meria ako kvadratická odchýlka, teda
$$\err(y, y') = (y - y')^2.$$



\subsection{Teoretické limity}
Najprv sa ale pozrieme na teoretické limity toho, ako dobrá vôbec
môže nejaká funkcia byť. Označme $h^{\square}$ najlepšiu možnú funkciu,
nemusí byť nutne z $H$. Teda
$$h^{\square} = \argmin_h \left( \err(h) \right) = \argmin_h \left( \E_{x,y} \left[ (h(x) - y)^2 \right] \right).$$
Jediné obmedzenia kladené na $h$ sú, že je to funkcia: pre každé $x$
musí vrátiť vždy jednu a tú istú hodnotu. Distribúcia $P$ ale nemusí
pre dané $x$ vždy vrátiť to isté $y$: môže byť zašumená, alebo
jednoducho $x$ neobsahuje dostatočnú informáciu. Napríklad, ak podľa
plochy bytu určujeme jeho cenu, niektoré dva byty môžu mať rovnakú
plochu a predsa rôznu cenu. Ako uvidíme, tento nedeterminizmus je
jediný dôvod, prečo hypotéza $h^{\square}$ nemusí mať nulovú chybu.

Chybu ľubovoľnej hypotézy $h$ vieme upraviť nasledovne:
\begin{align}
  \err(h)
    &= \E_{x,y} \left[ (h(x) - y)^2 \right] \\
    &= \E_{x} \left[ \E_{y|x} \left[ (h(x) - y)^2 \right] \right]
\end{align}
Pozrime sa na vnútornú strednú hodnotu. V nej je $x$ konštanta, a
teda aj $h(x) = c$ je konštanta. Aká konštanta minimalizuje danú
strednú hodnotu? Nie je ťažké vidieť (napríklad zderivovaním), že
minimum sa nadobúda pre $c = \E[y]$. Takže
$$h^{\square}(x) = \E_{y|x}[y],$$
a jeho priemerná chyba je
$$\err(h^{\square}) = \E_{x} \left[ \E_{y|x} \left[ (y - \E[y]) \right] \right] = \E_{x} \left[ \Var_{y|x}(y) \right].$$
Vidíme teda, že pokiaľ je $y$ jednoznačne určené $x$-om, tak
$h^{\square}$ bude mať nulovú chybu.



\subsection{Bias-variance tradeoff}
V tomto odseku si ukážeme zaujímavý výsledok, ktorý nám za určitých
predpokladov umožňuje vyjadriť chyby pomocou iných, jasnejších veličín:
tzv. \emph{výchylky} a \emph{rozptylu}.


\paragraph{Odvodenie.}
Označme najlepšiu hypotézu z množiny $H$ ako $h^\star$, teda
$$h^\star = \argmin_h \left( \err(h) \right).$$

Budeme upravovať výraz reprezentujúci priemernú chybu priemernej
hypotézy $\hat{h}$.
\begin{align}
  \chalg % \E_T \left[ \err(\hat{h}) \right]
    &= \E_T \left[ \err(\hat{h}) \right] \\
    &= \E_T \left[ \E_{x,y} \left[ (\hat{h}(x) - y)^2 \right] \right] \\
    &= \E_T \left[ \E_{x,y} \left[ \left( (\hat{h}(x) - h^\star(x)) + (h^\star(x) - y) \right)^2 \right] \right]
\end{align}
V tomto momente prichádza netriviálny technický krok, ktorý si
vyžaduje dodatočné predpoklady. Tieto technické detaily prenecháme
na koniec časti, sústreďme sa na to hlavné.
$$
  \chalg
    = \E_T \left[ \E_{x,y} \left[ (\hat{h}(x) - h^\star(x))^2 \right] \right]
    + \E_T \left[ \E_{x,y} \left[ (h^\star(x) - y)^2 \right] \right]
$$
Druhý zo sčítancov sa dá ešte zjednodušiť. Kedže $h^\star$ ani $y$
nezávisia od trénovacích dát, môžeme sa zbaviť vonkajšej strednej
hodnoty. Dostávame tak výslednú rovnosť
$$
  \chalg
    = \underbrace{\E_T \left[ \E_{x,y} \left[ (\hat{h}(x) - h^\star(x))^2 \right] \right]}_{\text{rozptyl}}
    + \underbrace{\E_{x,y} \left[ (h^\star(x) - y)^2 \right]}_{\text{výchylka}}.
$$

Prvý zo sčítancov budeme volať \emph{rozptyl}. Vyjadruje, ako ďaleko je naša
funkcia od najlepšej možnej, vrámci množiny hypotéz $H$. Druhý zo sčítancov
budeme volať \emph{výchylka}. Vyjadruje chybu, ktorá je spôsobená výberom
množiny hypotéz.

Výchylku vieme upraviť ďalej. Pretože hypotéza $h^\star$ ani $y$ nezávisia od
trénovacej množiny $T$, merať chybu na testovacích dátach $x, y$ je to
isté, ako merať ju na trénovacích dátach $x_i, y_i$, berúc ich náhodný
výber. Teda
\begin{align}
  \vychylka
    &= \E_T \left[ \E_{x_i, y_i} \left[ (h^\star(x_i) - y_i)^2 \right] \right] \\
    &= \E_T \left[ \E_{x_i, y_i} \left[ \left( (h^\star(x_i) - \hat{h}(x_i)) + (\hat{h}(x_i) - y_i) \right)^2 \right] \right]
\end{align}
Opäť, použitím toho istého technického kroku dostaneme:
\begin{align}
  \vychylka &= \underbrace{\E_T \left[ \E_{x_i, y_i} \left[ (h^\star(x_i) - \hat{h}(x_i))^2 \right] \right]}_{\text{trénovací rozptyl}}
    + \underbrace{\E_T \left[ \E_{x_i, y_i} \left[ (\hat{h}(x_i) - y_i)^2 \right] \right]}_{\text{priemerná trénovacia chyba}}
\end{align}

Trénovací rozptyl vyjadruje, ako ďaleko je naša hypotéza $\hat{h}$
od najlepšej možnej $h^\star$ z $H$. Na rozdiel od rozptylu ale túto
vzdialenosť meriame na trénovacích dátach, nie na testovacích. To spraví
rozdiel, nakoľko $\hat{h}$ je závislé od trénovacích dát.
Priemerná trénovacia chyba je priemerná chyba, ktorej sa dopustí výstup
z algoritmu $\hat{h}$ na tých istých dátach, pomocou ktorých sme $\hat{h}$
zostrojili.


\paragraph{Závery.}
Podarilo sa nám teda rozložiť chybu algoritmu na dve, prípadne tri časti.
Načo je to ale dobré? Ukážeme si, ako pomocou nich vieme získať intuíciu
o tom, ako sa správa chyba algoritmu v závislosti od veľkosti trénovacej
množiny a veľkosti (tj. zložitosti) množiny hypotéz.

\TODO{obrázok kriviek učenia, vysvetlenie}
\TODO{podučenie, preučenie}


\paragraph{Technické detaily.}
Nakoniec sa vyjadríme k spomínanému technickému kroku. Začneme jeho
znením a potom uvedieme jeho predpoklady.

\begin{theorem}
  Predpokladajme, že vstupom do hypotéz sú vektory reálnych čísel
  (tj. $X = \reals^n$), cieľom je predpovedať jedno reálne číslo
  (tj. $Y = \reals$), a že pravdepodobnostné rozdelenie $P$ je spojité.
  
  Nech množina hypotéz $H$ je uzavretá na lineárne kombinácie a na
  limity (teda ak postupnosť funkcii v $H$ konverguje, jej limita
  je tiež v $H$).
  
  Ďalej predpokladajme, že trénovací algoritmus vždy vráti takú funkciu
  $\hat{h} \in H$, ktorá minimalizuje trénovaciu chybu. Inak zapísané,
  $$\hat{h} = \argmin_{h \in H} \left( \E_T \left[ \err_T(h) \right] \right).$$
  
  Potom platí
  $$\E_T \left[ \E_{x,y} \left[ \left( (\hat{h}(x) - h^\star(x)) + (h^\star(x) - y) \right)^2 \right] \right] = \E_T \left[ \E_{x,y} \left[ (\hat{h}(x) - h^\star(x))^2 \right] \right] + \E_T \left[ \E_{x,y} \left[ (h^\star(x) - y)^2 \right] \right]$$
\end{theorem}
\begin{remark}
  Dokazovaná rovnosť je ekvivalentná s nasledovnou, stručnejšou:
  $$\E_T \left[ \E_{x, y} \left[ (\hat{h}(x) - h^\star(x)) \cdot (h^\star(x) - y) \right] \right] = 0.$$
  Túto kratšiu verziu získame roznásobením a použitím linearity strednej
  hodnoty. V dôkaze budeme dokazovať túto rovnosť.
\end{remark}
\begin{remark}
  Všimnite si, že potrebujeme uzavretosť množiny $H$ na limity na to,
  aby bolo $\argmin_{h \in H} (\ldots)$ dobre definované. Vo všeobecnosti
  nemusí existovať taká funkcia, ale môže existovať nekonečná postupnosť
  funkcii, každá ďalšia lepšia, ako tá predchádzajúca. (Inak povedané,
  neexistuje minimum, iba infimum.)
\end{remark}
\begin{remark}
  Veta by sa dala rozšíriť aj na iné množiny $X, Y$, napríklad keď
  predpovedaná premenná je vektor ($Y = \reals^m$), ... Možno ani $P$
  nemusí byť spojité. Pre jednoduchosť argumentu ale budeme uvažovať
  vetu tak, ako je popísaná vyššie.
\end{remark}
\begin{remark}
  Predpoklady vety sú značne obmedzujúce. Napríklad si uvedomte, že
  ju nie je možné použiť na klasifikáciu, či dokonca ani na ľubovoľnú
  ohraničenú regresiu (kde rozumné hodnoty $y$ sú ohraničené). Ale taká
  je teória.
\end{remark}

Pri našom dôkaze využijeme niekoľko vlastností funkcii, ktoré uvádzame
v nasledujúcom odseku. Skúsený čitateľ-matematik ho môže preskočiť.

\begin{definition}
  (Skalárny súčin.) Nech $f, g$ sú funkcie z $X$ do $\reals$,
  z nejakej príjemne sa správajúcej množiny funkcii (tj. rovnomerne
  spojité, ..., čokoľvek, aby nasledujúce argumenty prešli). Definujeme
  ich skalárny súčin $\langle \cdot, \cdot \rangle$ ako
  \begin{align}
    \langle f, g \rangle
      &= \int f(x) \cdot g(x)\ d\rho x \\
      &= \E_x \left[ f(x) \cdot g(x) \right],
  \end{align}
  kde $\rho$ je hustota pravdepodobnosti distribúcie $P$.
  Rozmyslite si, že takto definovaný skalárny súčin má všetky vlastnosti,
  ktoré sa bežne požadujú od skalárnych súčinov:
  \begin{itemize}
    \item Je symetrický od svojich argumentov, teda $\langle f, g \rangle = \langle g, f \rangle$.
    \item Je lineárny: $\langle f, g + h \rangle = \langle f, g \rangle + \langle f, h \rangle$
      a tiež $\langle k \cdot f, g \rangle = k \cdot \langle f, g \rangle$.
    \item $\langle f, f \rangle \geq 0$ pre ľubovoľné $f$, pričom rovnosť
      nastáva práve vtedy, keď je $f$ konštantne nulové.
  \end{itemize}
\end{definition}

\begin{definition}
  (Kolmosť.) Dve funkcie $f, g$ sú na seba kolmé, ak ich skalárny
  súčin je $0$. Značíme $f \perp g$.
\end{definition}

\begin{definition}
  (Norma.) Podľa skalárneho súčinu definujeme normu funkcie (jej ``dĺžku''):
  $$\norm{f} = \sqrt{\langle f, f \rangle} = \sqrt{\E_x \left[ f^2(x) \right] }$$
  Spĺňa \emph{trojuholníkovú nerovnosť}: pre ľubovoľné funkcie $f, g$
  platí
  $$\norm{f} + \norm{g} \geq \norm{f + g}.$$
  Definuje nám teda (euklidovskú) metriku nad funkciami, podľa
  ktorej definujeme limity a konvergenciu.
\end{definition}

\begin{lemma}
  (Pytagorova veta.) Nech $f \perp g$. Potom platí:
  $$\norm{f}^2 + \norm{g}^2 = \norm{f + g}^2$$
\end{lemma}
\begin{proof}
  Pozrime sa na pravú stranu. Iba v nej zapíšeme normu ako skalárny
  súčin a využijeme jeho linearitu a symetriu:
  \begin{align}
    \norm{f + g}^2
      &= \langle f + g, f + g \rangle \\
      &= \langle f, f \rangle + \langle g, g \rangle + 2 \cdot \langle f, g \rangle
  \end{align}
  Pretože $f \perp g$, posledný sčítanec je nulový, čím dostávame
  dokazované tvrdenie.
\end{proof}

\begin{definition}
  (Projekcia na množinu.) Nech $H$ je množina funkcii, ktorá je uzavretá
  na lineárne kombinácie a na limity, a nech $f$ je funkcia.
  \emph{Projekciu} funkcie $f$ na množinu $H$ budeme označovať $f_H$
  a budeme pod ňou rozumieť nasledovný výraz:
  $$f_H = \argmin_{h \in H} d(f, h)$$
\end{definition}

\begin{lemma}
  (Kolmosť projekcie.) Pre ľubovoľnú funkciu $h \in H$ platí $h \perp f - f_H$.
\end{lemma}
\begin{proof}
  Sporom, predpokladajme, že $h \not\perp f - f_H$. Takže
  $\langle h, f - f_H \rangle \neq 0$. Ukážeme, že potom existuje
  v $H$ funkcia, ktorá je k funkcii $f$ bližšie, ako funkcia $f_H$.
  To bude hľadaný spor s definíciou $f_H$.
  
  Pozrime sa na všetky funkcie, ktoré ležia na priamke
  $f_H + \Delta \cdot h$. Tieto funkci sú v množine $H$, pretože
  $f_H, h \in H$ a množina $H$ je uzavretá na lineárne kombinácie.
  Každú z týchto funkcii vieme asociovať s jedným reálnym číslom
  $\Delta$. Pozrime sa na ich vzdialenosti od funkcie $f$, vyjadrené
  ako funkcia od $\Delta$:
  \begin{align}
    \text{dist}(\Delta)
      &= d(f, f_H + \Delta \cdot h) \\
      &= \langle (f - f_H) + \Delta \cdot h, (f - f_H) + \Delta \cdot h \rangle \\
      &= \langle f - f_H, f - f_H \rangle + 2\Delta \cdot \langle h, f - f_H \rangle + \Delta^2 \cdot \langle h, h \rangle
  \end{align}
  Pozrime sa na deriváciu tejto funkcie. Podľa definície $f_H$ by
  malo byť $f - f_H$ najkratšie možné, teda pre $\Delta = 0$ by mala
  funkcia $\text{dist}$ nadobúdať minimum, a teda mať tam nulovú
  deriváciu. Uvidíme, že tomu tak nie je:
  \begin{align}
    \frac{\partial \text{dist}}{\partial \Delta} \left( 0 \right)
      &= \lim_{\Delta \to 0} \left( \frac{\text{dist}(\Delta) - \text{dist}(0)}{\Delta} \right) \\
      &= \lim_{\Delta \to 0} \left( \frac{2\Delta \cdot \langle h, f - f_H \rangle + \Delta^2 \cdot \langle h, h \rangle}{\Delta} \right) \\
      &= 2 \cdot \langle h, f - f_H \rangle
  \end{align}
  To je nenulové, nakoľko $h \not \perp f - f_H$. Čo je hľadaný spor.
\end{proof}

\TODO{dokončiť dôkaz}



\subsection{Bias-variance tradeoff, verzia 2.}
V literatúre pod názvom \emph{bias-variance tradeoff} vystupuje aj
podobný, ale predsa odlišný výsledok, ako bolo uvedené vyššie.
Ukážeme a odvodíme si ho.

\begin{theorem}
  Nech $y : X \to \reals$ je funkcia, ktorú sa snažíme modelovať.
  Predpokladajme, že sa dá rozložiť na časti: $y = f(x) + \varepsilon$,
  kde $\varepsilon$ hrá rolu šumu: je nezávislý od všetkého a
  $\E[\varepsilon] = 0$. Označíme jeho pravdepodobnostnú distribúciu
  $E$.
  
  Nech výstupom trénovacieho algoritmu je $\hat{f}$. Za chybovú
  funkciu zvoľme kvadratickú chybu. Chybu algoritmu vieme teda
  vypočítať nasledovne:
  $$\chalg = \E_{(x, y) \sim P, T \sim P^t, \varepsilon \sim E} \left[ (\hat{f}(x) - y)^2 \right].$$
  
  Tvrdíme, že sa dá rozložiť na tri nasledovné časti:
  $$
  \chalg
      = \underbrace{\Var(\hat{f}(x) - f(x))}_{\text{\normalfont rozptyl}}
      + \underbrace{(\E[\hat{f}(x)] - \E[f(x)])^2}_{\text{\normalfont výchylka}^2}
      + \underbrace{\Var(\varepsilon)}_{\text{\normalfont šum}}
  $$
\end{theorem}
\begin{remark}
  V poslednej rovnici sme kvôli stručnosti vynechali pri stredných
  hodnotách a rozptyloch premenné a distribúcie, z ktorých ich berieme.
  V dôkaze budeme vždy brať všetky premenné z ich príslušných distribúcii.
\end{remark}
\begin{remark}
  Funkcia $f$ hrá v podstate tú istú rolu, čo najlepšia možná hypotéza
  spomedzi všetkých funkcii (nielen tých v množine hypotéz), $h^\square$.
\end{remark}
\begin{remark}
  V tomto znení bias-variance tradeoff-u názvy \emph{rozptyl} a
  \emph{výchylka} zodpovedajú príslušným štatistickým/pravdepodobnostným
  pojmom.
\end{remark}
\begin{remark}
  Na rozdiel od predchádzajúcej verzie bias-variance tradeoff-u, tu
  nebudeme potrebovať žiadne dodatočné predpoklady od algoritmu ani
  od jeho množiny hypotéz. (Nemusí teda vracať hypotézu, ktorá je
  spomedzi hypotéz v $H$ najlepšia na daných trénovacích dátach.
  Takisto od množiny hypotéz nepožadujeme žiadne vlastnosti.)
\end{remark}
\begin{proof}
  Upravujme pôvodný výraz.
  \begin{align}
    \chalg
      &= \E \left[ (\hat{f}(x) - y)^2 \right] \\
      &= \E \left[ (\hat{f}(x) - f(x) - \varepsilon)^2 \right] \\
      &= \E \left[ (\hat{f}(x) - f(x))^2 \right] + \E \left[ \varepsilon^2 \right] - 2 \cdot \E \left[ \varepsilon \cdot (\hat{f}(x) - f(x)) \right] \\
      &= \E \left[ (\hat{f}(x) - f(x))^2 \right] + \E \left[ \varepsilon^2 \right]
  \end{align}
  Výraz sme upravili, roznásobili a využili linearitu strednej hodnoty.
  V poslednom kroku sme použili $\E[ab] = \E[a] \cdot \E[b]$, ktorý
  platí pre ľubovoľné nezávislé premenné, s $a := \varepsilon$,
  $b := \hat{f}(x) - f(x)$. Zamerajme sa ďalej na prvý sčítanec.
  \begin{align}
    \text{prvý sčítanec}
      &= \E \left[ (\hat{f}(x) - f(x))^2 \right] \\
      &= \E[\hat{f}(x)^2] + \E[f(x)^2] - 2 \cdot \E[\hat{f}(x) \cdot f(x)] \\
      &= (\Var(\hat{f}(x)) + \E[\hat{f}(x)]^2) + (\Var(f(x)) + \E[f(x)]^2) - 2 \cdot \E[\hat{f}(x) \cdot f(x)]
  \end{align}
  V poslednom kroku sme využili vzťah $\Var(a) = \E[a^2] - \E[a]^2$.
  Pokračujme ďalej v úpravách.
  \begin{align}
    \begin{split}
      \text{prvý sčítanec}
        &= \Var(\hat{f}(x)) + \Var(f(x)) + (\E[\hat{f}(x)] - \E[f(x)])^2 \\
        &\quad + 2 \cdot \E[\hat{f}(x)] \cdot \E[f(x)] - 2 \cdot \E[\hat{f}(x) \cdot f(x)]
    \end{split} \\
    &= \Var(\hat{f}(x)) + \Var(f(x)) + (\E[\hat{f}(x)] - \E[f(x)])^2 - 2 \cdot \Cov(\hat{f}(x), f(x)) \\
    &= \Var(\hat{f}(x) - f(x)) + (\E[\hat{f}(x)] - \E[f(x)])^2
  \end{align}
  Využili sme najprv vzťah $\Cov(a, b) = \E[ab] - \E[a] \cdot \E[b]$,
  a potom $\Var(a - b) = \Var(a) + \Var(b) - 2 \cdot \Cov(a, b)$.
  Keď to teda celé dáme do jednej rovnice, dostaneme
  $$
  \chalg
      = \underbrace{\Var(\hat{f}(x) - f(x))}_{\text{\normalfont rozptyl}}
      + \underbrace{(\E[\hat{f}(x)] - \E[f(x)])^2}_{\text{\normalfont výchylka}^2}
      + \underbrace{\Var(\varepsilon)}_{\text{\normalfont šum}}
  $$
\end{proof}




\section{Ako sa vysporiadať s preučením/podučením?}

\TODO{regularizácia}
\TODO{holdout testing}
\TODO{$k$-fold cross validation}
\TODO{best practices}
